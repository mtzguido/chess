\documentclass{article}
\usepackage[utf8]{inputenc}
\usepackage{fullpage}

\renewcommand{\refname}{\section{Referencias}}

\begin{document}

\title{Proyecto IIA - Ajedrez}
\date{11 de Marzo, 2014}

\author{Guido Martínez \\ LCC, FCEIA, UNR }

\maketitle

\section{Objetivo}

Crear una inteligencia artificial para jugar al ajedrez, usando métodos tradicionales. Priorizar la eficiencia ante la claridad del código, y priorizar la calidad de cada movida al tiempo de ejecución (mientras sea razonable).

\section{Implementación}

El algoritmo de búsqueda elegido es Minimax, con las mejoras de poda alfa-beta y {\it Killer Heuristic}.
La profundidad usada es de 6  {\it plies} (movidas de algún jugador).
Está profundidad máxima se extiende en el caso de que la última movida sea una captura, una promoción, o un jaque.
Esto ayuda a evitar el efecto horizonte, porque tendemos a no dejar de buscar si la última movida fue {\it ruidosa}. 
El algoritmo es {\bf determinista} y no tiene en cuenta el tiempo que tarda.\\

La función de evaluación del tablero combina:
\begin{itemize}
\item Puntaje de fichas individuales
\item Puntaje de fichas por posición
\item Jaques actuales
\end{itemize}

Las tablas ficha-posición fueron tomadas de \cite{piece-square-table} y ligeramente modificadas.\\

La representación del tablero se hizo de cero e incluye muchos datos sobre el tablero, que se calculan incrementalmente en vez de calcularse nuevamente para cada tablero, esto permite calcular el puntaje de las fichas muy rápidamente.
También, al ver si un tablero está en jaque, se guarda el resultado en la misma estructura para no recalcularlo varias veces.
Al hacer una movida, se decide si la jugada puede haber influenciado la amenaza al rey, y en caso negativo, no se recalcula el jaque.

\section{Resultados}
Probando nuestra IA contra otras ya existentes, notamos muy buenos resultados. A continuación, algunos resultados:

\begin{itemize}
\item Nombre IA: \hfill x/y/z, porcentajes de ganados, tablas, perdidos
\item Contra Fairy (IA por defecto de xboard): \hfill ?/?/?
\end{itemize}


\begin{thebibliography}{9}

\bibitem{piece-square-table}
  http://chessprogramming.wikispaces.com/Simplified+evaluation+function

\end{thebibliography}



\end{document}
